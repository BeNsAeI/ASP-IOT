\subsubsection{Fall 2018}
            \textbf{Week 4}
            \begin{itemize}
                \item We had a final draft of the problem statement as a group due this week. We all met up a few times in the library to work on it together. Things went well. Mehgan and I’s writing went pretty well together because we addressed different aspects of the problem statement, so we were able to combine with Louis’ material to create something satisfactory. We finished, and a group member turned it in. We also met with our sponsor Ben and Jenna did not attend however that was not a issue. As a group we went over the project requirement assignment and asked Ben questions we had. The meeting went well. Ben is going to ask for some hardware, so we can start tinkering. We are planning on starting the Mozilla A-frame tutorial to play around and see how it works. Once we have the hardware a rasp pi we are going to mount a camera on it and try to make it 3D and see if the rasp pi can handle it. If not, then we are going to look at the beagle bone instead. No problems everything is going well so far.
            \end{itemize}
            
            \textbf{Week 5}
            \begin{itemize}
                \item We all worked on system requirement document together. Then we also met with both our TA and client. In our meeting with Ben we got feedback for the requirements doc because we were missing a few things. We are planning on finished that up.
            \end{itemize}
            
            \textbf{Week 6}
            \begin{itemize}
                \item We have been meeting with our client regularly so far. When we met with him Ben gave us some pointers on what he would like to see changed on the system requirements document. As a team we then discussed it and worked on the document. We finished it and turned it in. The only issue we have had recently was that we still had not gotten our hands on any hardware. This week after our meeting with Ben he was able to get a rasp pi from Kevin to us. Plans for next week is focusing on a paper prototype. We will come together to discuss it and draw it up. After the light weight prototype is done we start testing the hardware to see if it can stream video at a decent rate.
            \end{itemize}
            \newpage
            \textbf{Week 7}
            \begin{itemize}
                \item The main goals of this week in class was to finish the tech review. A goal for the project we had as a group was to do a paper prototype. We met as a group and worked on it together and got it done. We were able to meet with our sponsor Ben. Jenna was not present but joined us briefly over speakerphone. We went over our paper prototype in the meeting and got pointers on what to add and edit. Ben looked over the system requirements doc and also gave us more pointers. There have been no problems. The goal for next week it to deliver to our client a lightweight prototype that will be a nicer digital copy of our paper prototype.
            \end{itemize}
            
            \textbf{Week 8}
            \begin{itemize}
                \item We went over the lightweight prototype with our client and took note of his feedback in regards to certain aspects we need to change. The currently problem is that we are not streaming yet.
            \end{itemize}
            
            \textbf{Week 9}
            \begin{itemize}
                \item We didn't meet but have still not been able to get the pi to stream video. We are going to work more on the project in winter term after we are able to get more hardware.
            \end{itemize}
    
        \subsubsection{Winter 2019}
            \textbf{Week 1}
            \begin{itemize}
                \item This week our team talked about our schedules and found times that we can meet. Also progress on the website was done! The plan is to begin working on hardware. Specifically get the camera streaming regular video. There haven't been any problem and we are planing on meeting with our sponsor next week. 
            \end{itemize}
            
            \textbf{Week 2}
            \begin{itemize}
                \item This week we fixed some details on the website. Also we talked to our sponsor and set up a time that work for everyone to meet. We met as a team without our sponsor to plan what we will each do. We missed our meeting with the TA because we could not find her. After our group meeting we emailed her telling her our plans. I’m going to work on getting the camera streaming.  Louis will work on the website, and Meghan on the 360 camera 
            \end{itemize}
            
            \textbf{Week 3}
            \begin{itemize}
                \item We were able to set a meeting time with our Sponsor. We met and showed both Ben and Jenna the website. We talked about our plans. I am going to work with the raspberry pi, Meghan on the 3D camera, and Louis on the website some more. There are no problems. We were also able to meet with our TA. Last week we had an issue meeting and it was resolved so we have a set time and location for meeting. We also worked on the poster together and wrote an elevator speech for the project.
            \end{itemize}
            \newpage
            \textbf{Week 4}
            \begin{itemize}
                \item Louis who is working on the website passed the raspiberry pi B to me. He also mention tutorials that didn't work for streaming and pointed me in the right direction for another to try. I tired streaming with VLC media steamer. I two types of tutorials TCP and UDP however none of them worked. However it should have. My main issue was getting it to stream. Then I noticed I couldn't access the internet. It said that the raspi was connected to the the network but I couldn't visit websites. I looked up the error the web browser gave me and it said that the clock was out of sync with the time so it could not access the internet. I know my apartments internet has multistep authentication so I can access the network. The raspi cannot get online to update its time to be able to browse the internet. I learned how to update the rasps clock manually through the settings configuration. I was able to fixed the clock thinking that it would solve the issue with streaming but it did not. I continued to look for a solution until I found a promising tutorial. I passed the tutorial and pi onto Meghan to try on her wifi. Plans are to continue working with on getting video streaming working. 
            \end{itemize}
            
            \textbf{Week 5}
            \begin{itemize}
                \item Louis recommended me another tutorial to try. He has been working on the website so I tried a tutorial that uses a VLC media player. I was able to capture video with the raspi and display it to a connected monitor. Next, I attempted to stream video from the raspi to a computer on the same network but I was not able to get it to work.  I taught Meghan how to use a raspberry pi for the first time. I walked her through the tutorial I used to get it to capture video. After I did more research, I found that all I was missing was to give the stream a name. I explained to Meghan what was missing and she tried it and it worked! Louis added functionality to the grid to add devices on the map for the website. The next step for me is to work on streaming from computer on different networks. I will work on that with Meghan at her house because we've had a difficult time with the multi factor authentication needed for higher security networks which is the case in for OSU and my home wifi. Also I will be asking Ben for the nVidia Jetson, and am so excited to work with it!
            \end{itemize}
            
            \textbf{Week 6}
            \begin{itemize}
                \item We still have the issue of not being able to stream on the schools wifi because of the firewalls however we can stream on Meghans home wifi. We tried to stream from different networks but still can do it. We set up port forwarding on her router and it did not work. We are still trying to stream from different networks. We did however take a video of streaming from within the same network and sent it to our sonar Ben. We also asked for more hardware from both Ben and Kevin. We will be getting another raspi but a better model than the one we have. We expect the video stream quality to improve with the newer model. Also we will be getting a Jetson TX2 to start streaming to. Also from Ben we will get a microphone and start to start streaming audio. There has been some more progress on the website where devices can be added onto the map. Once we can stream to the website then we will add full functionality to that feature.
            \end{itemize}
            
            \textbf{Week 7}
            \begin{itemize}
                \item This week we met with Ben and discussed our current problem. We haven’t been able to stream from different networks. He gave us a few different methods besides what we thought to do. We plan on still trying packet forwarding again. Also another  idea is to get a domain name. Also we plan on starting to stream audio.  
            \end{itemize}
            
            \textbf{Week 8}
            \begin{itemize}
                \item Since we hadn't be able to get packet forwarding working we tried some of the methods Ben gave us. Originally when I was working with the raspi at Meghan's the tutorials I was following and had then showed her only had us set up packet forwarding with one port. Then after trying to use multiple ports it worked. Streaming regular video is a big part of our project after achieving that we talked about what we'll do next. I plan on working on streaming audio next week. There are currently no problems. 
            \end{itemize}
            
            \textbf{Week 9}
            \begin{itemize}
                \item I emailed the client  a link to a micro phone for him tp purchase. A few days afterward at our meeting he gave us hardware for a second raspberry pi but newer model the B+. Meghan had been using the first one after I worked on it because she had wifi that worker better for us. I got the rasp pi B and Meghan the new one. I worked plugged in the microphone and tested it. Tried some tutorials and learned how to record sound and play it back. I also tired to stream audio. I have an error that the mux isn't specified. In the tutorial I'm using VLC as my medias streamer. That is my current issue. For the future I plan on trying to stream audio with a different format. Right now I only have a usb microphone and not the camera module. We will also try another method of streaming audio and video at the same time. Do this may save us the work of over laying audio onto the video. 
            \end{itemize}
            
            \textbf{Week 10}
            \begin{itemize}
                \item This week we worked together on updating the report based on feedback and the video. We meet up at the library and made slides for the videos. We wrote out a script and agreed on who’s doing what slide. We also talked about how we would do the video. The plan is to meet up to record the voice over of the script, then editing that on a video showing google slides and demoing the system. We also met with our client and talked about getting the Jetson next term after we test whether a newer model the raspberry pi B+ can handle 360 video. No problems this week.  
            \end{itemize}
            
        \subsubsection{Spring 2019}
            \textbf{Week 1}
            \begin{itemize}
                \item We met as a team and talk about what we did over spring break and made plans on how to move forward. Over spring break, I worked on the manual. This week we focused on audio. I got access to a room in Kelley where we can set things up and make our regular meeting time. I registered us for expo. I got the model release form and had my teammates fill them out. I also worked on the audio. I mention to the team how previously I was working on getting the audio working separately and was going o try to make it work in conjunction with the video. I met with Meghan and work on it. An issue we had was with a new raspi api called picam was not working. It gave us licensing errors and as soon as we fixed one, we would get a different error. Because we were on an errors goose chase, we stopped with picam and looked for other solutions. We had tried another tutorial and did not work this went on for a few more tutorials be finally finding one that streams both audio and video. The next step is to implement this to work across networks and integrate it to the website. Also, next week we will be focusing more on 360 implementation and setting up an ad hoc network. We still need more hardware to work on he ad hoc network and I have requested that from Ben.
            \end{itemize}
            \newpage
            \textbf{Week 2}
            \begin{itemize}
                \item This week we were able to meet with our TA as a group. I met with Meghan at her house to work on streaming audio. We were stuck on being able to stream audio but finally go it working it. We learned that the raspberry pi that her router is port forwarding to has to be the receiver. When we had tried the raspberry pi that is connected to a separate network to receive from the raspi on her home network it had not work. So now we are streaming audio from different networks. One issue we have is that the microphone sounds absolutely & completely terrible. I brought my own personal high end streaming blue yeti microphone to test the differences in the sound quality and the higher end microphone is better than the inexpensive USB microphone we are using for our system. I looked up on how fix the audio with our project microphone and played with different sample rates. The higher the sample rate the better it sounded. So, we were able to get it some completely not understandable to kind of bad. We met with our client and played him two audio clips from the different microphones, and I believe we will be buying a better microphone. Next we tried to stream audio through the server and currently it is making the file but not putting anything in it. We are going to get audio working on the server then overlay the audio with the video which we are already streaming
            \end{itemize}
            
            \textbf{Week 3}
            \begin{itemize}
                \item This week we worked on our project to get it ready for code freeze. We worked at Meghan’s house to get audio working and finally did. We worked on the installation and operation manuals for code freeze. We got the project to stream audio and video the two major requirements but did not accomplish 360 streaming by code freeze. Originally we had planned to use a Jetson then Ben asked us to  use on raspberry pi’s to keep the cost down. When we could not get A-frame to stream our video as 360 we were stuff because it should be able to stream the format we use for normal video. When it did not work we tried to convert the video format from hls to mp4 however the raspi we used for the conversion frozen and could not handle it. I talked to our Client Ben individually to ask for the Jetson and explained why we needed it. He said that a-frame should stream our format and that while he could maybe get us the jetson it would cost us a lot of overhead time setting it up so he rather we not use one. After code freeze we worked on updating the design and requirements documents and has Ben sign them. Then we figured out what we were missing to stream 360 video. Now streaming audio, normal, and 360 video we have met the features. The next step is to unwrap the 360 video basically just formatting it so it looks nice. Then lastly will be 3D printing a few pieces to mount onto the raspis and testing batteries.
            \end{itemize}
            
            \textbf{Week 4}
            \begin{itemize}
                \item This week asked Ben to print out a 3D case for the pi cam that the 360 video lens will be attached to. After it got printed, I picked it up and took it to Meghan where we then attached things. We also worked on this week’s assignment which is the poster. We removed mentions about the ad hoc and changed system overview picture to a UML diagram. We met with Ben and went over the poster. Our client gave us some areas to revise and we did so. We then turned in the poster. Also, this week we worked a bit with a frame and now are viewing the 360 videos. Plans for next week is just unwrap the video and we are done.  
            \end{itemize}
            \newpage
            \textbf{Week 5}
            \begin{itemize}
                \item This we our everyone in the team had midterms and there were at conflicting times so one person had them at the start another in the middle and the other at the and so meeting was difficult because we were all busy. We did meet as a team though and talked about dewrapping the 360 video. We looked up tutorials and found a starting point. We are not further ahead because the problem is that while aframe gives variables and such that we should be able to use to adjust the center and radius to dewrap it does not actually have an documentation on how to do it then searching online for tutorials does not give a lot. We had a starting point and more time to do it next week so the goal is to get it done before friday next week so we can show our client. This week we also finished up our final draft of the poster with feedback from Ben and Kirsten and sent that out to print.
            \end{itemize}
            
           \textbf{Week 6}
            \begin{itemize}
                \item One week before expo we finished up our project. We had to finish up our last feature which was 360 video. First we installed a simple CV on the pi then followed a tutorial to dewrap the images and then after dewrappping we have to put the images together into video then server that on the website. So we got it done but we met with our client Ben and told him that this process we are doing to get the 360 is so extremely slow. He suggested things to do to optimize the code and try running it using python 3 instead of 2.7. We plan on meeting as a group on Monday to time the processing, cut non essential parts of the code out and see if we can make things faster.

            \end{itemize}