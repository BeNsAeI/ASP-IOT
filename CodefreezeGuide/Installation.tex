\documentclass[onecolumn, draftclsnofoot,10pt, compsoc]{IEEEtran}
\usepackage{graphicx}
\usepackage{url}
\usepackage{setspace}
\usepackage{listings}


\usepackage{geometry}
\geometry{textheight=9.5in, textwidth=7in}

% 1. Fill in these details
\def \CapstoneTeamName{		Nexusphere}
\def \CapstoneTeamNumber{		48}
\def \GroupMemberOne{			Meghan Mowery}
\def \GroupMemberTwo{			Louis Duvoisin}
\def \GroupMemberThree{			Sarahi Pelayo}
\def \CapstoneProjectName{		A-Frame Live Stream Portal}
\def \CapstoneSponsorCompany{	Oregon State University}
\def \CapstoneSponsorPerson{		Behnam Saeedi}

% 2. Uncomment the appropriate line below so that the document type works
\def \DocType{		Installation Guide
				%Requirements Document
				%Technology Review
				%Design Document
				%Progress Report
				}
			
\newcommand{\NameSigPair}[1]{\par
\makebox[2.75in][r]{#1} \hfil 	\makebox[3.25in]{\makebox[2.25in]{\hrulefill} \hfill		\makebox[.75in]{\hrulefill}}
\par\vspace{-12pt} \textit{\tiny\noindent
\makebox[2.75in]{} \hfil		\makebox[3.25in]{\makebox[2.25in][r]{Signature} \hfill	\makebox[.75in][r]{Date}}}}
% 3. If the document is not to be signed, uncomment the RENEWcommand below
%\renewcommand{\NameSigPair}[1]{#1}

%%%%%%%%%%%%%%%%%%%%%%%%%%%%%%%%%%%%%%%
\begin{document}
\begin{titlepage}
    \pagenumbering{gobble}
    \begin{singlespace}
    	%\includegraphics[height=4cm]{coe_v_spot1}
        \hfill 
        % 4. If you have a logo, use this includegraphics command to put it on the coversheet.
        %\includegraphics[height=4cm]{CompanyLogo}   
        \par\vspace{.2in}
        \centering
        \scshape{
            \huge CS Capstone \DocType \par
            {\large\today}\par
            \vspace{.5in}
            \textbf{\Huge\CapstoneProjectName}\par
            \vfill
            {\large Prepared for}\par
            \Huge \CapstoneSponsorCompany\par
            \vspace{5pt}
            {\Large\CapstoneSponsorPerson\par}
            {\large Prepared by }\par
            Group\CapstoneTeamNumber\par
            % 5. comment out the line below this one if you do not wish to name your team
            \CapstoneTeamName\par 
            \vspace{5pt}
            {\Large
                \GroupMemberOne\par
                \GroupMemberTwo\par
                \GroupMemberThree\par
            }
            \vspace{20pt}
        }
       % \pagebreak
        \begin{abstract}
        % 6. Fill in your abstract    
        	This is our installation guide for our A-Frame Live Stream Portal project. This document lists the hardware and software that will be needed. It also describes how to install the software on to the hardware. The installation guide does not go into detail about how to use the system because it is out of this documents scope.
        \end{abstract}     
    \end{singlespace}
\end{titlepage}
\newpage
\pagenumbering{arabic}
\tableofcontents
% 7. uncomment this (if applicable). Consider adding a page break.
%\listoffigures
%\listoftables
\clearpage

% 8. now you write!
\section{Materials}
Materials that we used for this project are as follows: \\
\begin{enumerate}
    \item Raspberry pi 3 b+
    \item Raspberry pi camera module
    \item 1 monitor
    \item 1 mouse
    \item 1 microphone
    \item 1 keyboard
    \item HDMI cable
    \item Power cable
    \item 360 lens
\end{enumerate}
\begin{itemize}
    \item Note: for this project we used the Kogeto Dot 360 lens for the iPhone.
\end{itemize}

\section{Raspberry Pi setup}
\subsection{The Pi Itself}
\begin{enumerate}

\item Install whatever version of raspbian you would like on your raspberry pi (for this project we used raspbian stretch), more infomation can be found here:  https://www.raspberrypi.org/downloads/raspbian/
\item Plug it into the peripherals (monitor, keyboard, mouse, etc...).
\begin{itemize}
    \item Note: you may need to plug in the peripherals before turning on the raspberry pi.
\end{itemize}
\item Follow the setup that the raspberry pi asks for, set the time zone and make sure you are connected to the internet, then update the pi.
\item Find the IP address of the pi by either running 'ifconfig' on the pi's terminal or hovering over the internet connection symbol on the top right of the screen.
\item Run 'sudo apt-get install npm'.
\item Set the raspberry pi to receive video
\begin{enumerate}

    \item enter: 'sudo raspi-config' into the command line
    \item select 'Interfacing options'
    \item select 'camera'
    \item select 'yes'
    \item select 'OK'
    \item select 'Finish'
    \item select 'reboot' 
    \end{enumerate}

\item The pi will then restart with the camera option enabled. 
\end{enumerate}

\subsubsection{Installing picam}
\begin{enumerate}
\item  Run: sudo apt-get update
\item  Run: sudo apt-get install libharfbuzz0b libfontconfig1 
\item  From the 'Script' directory in the GitHub repository, move the 'make\_dirsh.sh' script to your pi's main page.
\item  Note: or you can enter it manually by entering: 
\begin{lstlisting}
cat > make_dirs.sh <<'EOF' 
#!/bin/bash 
DEST_DIR=~/picam 
SHM_DIR=/run/shm 

mkdir -p $SHM_DIR/rec 
mkdir -p $SHM_DIR/hooks 
mkdir -p $SHM_DIR/state 
mkdir -p $DEST_DIR/archive 

ln -sfn $DEST_DIR/archive  $SHM_DIR/rec/archive 
ln -sfn $SHM_DIR/rec  $DEST_DIR/rec 
ln -sfn $SHM_DIR/hooks $DEST_DIR/hooks 
ln -sfn $SHM_DIR/state $DEST_DIR/state 
EOF
\end{lstlisting}

\item  Run: chmod +x make\_dirs.sh to add permissions to run the script. 
\item  Run the script: ./make\_dirs.sh  
\begin{itemize}
    \item Note: if you want to adjust the sound run: 'alsamixer'
\end{itemize}
\item  Run:
wget https://github.com/iizukanao/picam/releases/download/v1.4.7/picam-1.4.7-binary.tar.xz 
\item  Enter: tar xvf picam-1.4.7-binary.tar.xz 
\item  Enter: cp picam-1.4.7-binary/picam \~{}/picam/ 
\item  Enter: 'cd \~{}/picam' to move to the picam directory. 
\item  To test that the camera is running correctly, run: ./picam --alsadev hw:1,0 and you should see these lines:\\
configuring devices\\
capturing started
\item  Return to the home directory before moving to the next section.
\begin{itemize}
    \item Note: If you experience any errors, check that all lines of code were typed correctly without any spelling or grammatical errors.
\end{itemize}

\end{enumerate}
\subsubsection{Setting up picam server}
\begin{enumerate}
\item  Enter: sudo npm install coffee-script -g 
\item  Enter: git clone https://github.com/iizukanao/node-rtsp-rtmp-server.git 
\item  Enter: cd node-rtsp-rtmp-server 
\item  Enter: npm install -d 
\item  Enter: cd \~{}/node-rtsp-rtmp-server
\item  In http.coffee, add 'header += "Access-Control-Allow-Origin: *\textbackslash n" ' under line 108. 
\begin{itemize}
    \item Note: Make sure that port forwarding is properly set up between your pi and the router before you begin streaming, check section 2.3 for details.
\end{itemize}

\end{enumerate}
\subsection{Video streaming}
\begin{enumerate}
    \item In one terminal enter:
    \begin{itemize}
        \item cd \~{}/node-rtsp-rtmp-server 
        \item ./start\_server.sh & 
    \end{itemize}
\item In a second terminal enter: 
\begin{itemize}
    \item  cd \~{}/picam 
    \item mkdir \~{}/node-rtsp-rtmp-server/public/picam/
    \item mkdir \~{}/node-rtsp-rtmp-server/public/picam/stream
    \item ./picam --alsadev hw:1,0 --rtspout -w 800 -h 480 -v 500000 -f 20 --hlsdir \~{}/node-rtsp-rtmp-server/public/picam/stream & 
    \item Note: If you have errors when starting the stream, make sure that the camera module is properly plugged into the raspberry pi and the pi is configured to receive video (see section 6 in 2.1 for more details).
\end{itemize}
\end{enumerate}

\subsection{Port forwarding}
The node server on the pi expects to get requests on port 80, which is troublesome if you have multiple Pis streaming. You'll need to open the router settings on the network where you plan to be streaming from. Typically the IP for this will be 192.168.1.1 but this may change router to router.

\begin{enumerate}
    \item On the Raspberry Pi type 'ifconfig' in a terminal window and find the IP address starting with 192 (this is the IP of the pi on the local network).
    \item  In the router settings forward whatever external port you want (typically above 1023) to port 80 of the Raspberry Pi's IP. Look up how to do this for your router as it varies make to make.
    \item  Go to a computer and type "what is my ip" in to google and get your ipv4 address, write it and the external port you chose in step 2 down, we'll need them later to add a device.
\end{enumerate}


\section{Server}
\subsection{Upload files}
\begin{enumerate}
    \item First, download all of the code from our GitHub repository to your own computer: https://github.com/BeNsAeI/ASP-IOT
    
    \item Put the Code directory in your public\_html directory on the OSU servers.
    \begin{itemize}
        \item Note: the files under the 'Code' directory will be used for the website and the 'Script' directory will be used to execute the stream commands.
    \end{itemize}
    \item Put the following bash script from the 'Script' directory (named permissions.sh) in the Code directory in your public\_html if you don't want to change every permission manually. Make sure you change the permissions of the script itself in order to run it. You can also enter it manually:
\end{enumerate}

\begin{lstlisting}
#!/bin/bash
chmod 751 *.php
chmod 771 database.php
chmod 775 css
chmod 775 css/*
chmod 775 images
chmod 775 images/*
chmod 775 js
chmod 775 js/*
chmod 771 errors
chmod 771 errors/*
\end{lstlisting}
You may see some errors about certain files not existing (particularly errors/*) these will go away as you use the website.

\subsection{Create a database}
Note: if you already have your own database you can simply import the sample-db.sql file and skip this step. 
\begin{enumerate}
    \item Set up your ONID database, log in to onid.oregonstate.edu with your ONID login information. 
    \item Click on the "Web Database" link on the left side of the page, create the database if needed. 
    \item Write down or take a picture of the database settings table, we'll need it in a future step. (You don't need to write it down if you just keep the tab open). 
    \item Open PHPMyAdmin here: https://secure.oregonstate.edu/oniddb and login with the database username and database password (NOT ONID). 
    \item Click on the current database link on the left side of the page (should be your database username). 
    \item Click the import tab and upload the 'sample-db.sql' file found in the main page of the GitHub repository. 
    \item Click on either the devices or map tables on the left just to check if they exist and have data.
\end{enumerate}
\subsection{Additional code}
\begin{enumerate}
    \item Create a file called config.php in your Code directory. 
    \item Enter the following code in to config.php (don't copy paste, some characters like quotes and dashes will break):

    \begin{lstlisting}
    <?php 
    global $databasepass;
    $databasepass = 'XXX'; 
    global $databasehost;
    $databasehost = 'XXX';
    global $databasename;
    $databasename = 'XXX';
    global $databaseuser;
    $databaseuser = 'XXX';
    ?>
    \end{lstlisting}
    \begin{itemize}
        \item You'll need to enter your own database password, host, name, and user from the previous section in to the respective fields.
        \item Note: follow section 3.2 if you don't have a database setup
    \end{itemize}
    \item After the file is written and saved, run the bash script (e.g. permissions.sh) again to ensure all the files have proper permissions. 
    \item The webpage should now be viewable on http://web.engr.oregonstate.edu/\textasciitilde'ONIDUSERNAME'/Code/index.php 
    \begin{itemize}
        \item If you downloaded the GitHub repository directly to the server the url might be /ASP-IOT/Code/index.php instead.
    \end{itemize}
    \item Log in using either the "admin" or "viewer" tokens.
\end{enumerate}
\subsection{Creating the device}
\begin{enumerate}
    \item Once everything is installed and streaming, you can add a device to the website.
    \item  Go to the website, and log in as an admin using the "admin" token.
    \item Click the add device button, pick a name (alphanumeric), code (alphanumeric), and map coordinates (1 indexed) for the device.
    \item Enter the IP and Port of the raspberry pi. Check the port forwarding section for more information.
    \item Click on the device you just made and watch the stream.
    \item Note: adding a device to the website is gone over in detail in the 'Operation' manual. 
\end{enumerate}

\section{Conclusion}
This concludes the instillation for the A-frame live stream portal project.
\end{document}
