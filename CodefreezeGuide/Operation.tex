\documentclass[onecolumn, draftclsnofoot,10pt, compsoc]{IEEEtran}
\usepackage{graphicx}
\usepackage{url}
\usepackage{setspace}

\usepackage{geometry}
\geometry{textheight=9.5in, textwidth=7in}

% 1. Fill in these details
\def \CapstoneTeamName{		Nexusphere}
\def \CapstoneTeamNumber{		48}
\def \GroupMemberOne{			Meghan Mowery}
\def \GroupMemberTwo{			Louis Duvoisin}
\def \GroupMemberThree{			Sarahi Pelayo}
\def \CapstoneProjectName{		A-Frame Live Stream Portal}
\def \CapstoneSponsorCompany{	Oregon State University}
\def \CapstoneSponsorPerson{		Behnam Saeedi}

% 2. Uncomment the appropriate line below so that the document type works
\def \DocType{		Operation Guide
				%Requirements Document
				%Technology Review
				%Design Document
				%Progress Report
				}
			
\newcommand{\NameSigPair}[1]{\par
\makebox[2.75in][r]{#1} \hfil 	\makebox[3.25in]{\makebox[2.25in]{\hrulefill} \hfill		\makebox[.75in]{\hrulefill}}
\par\vspace{-12pt} \textit{\tiny\noindent
\makebox[2.75in]{} \hfil		\makebox[3.25in]{\makebox[2.25in][r]{Signature} \hfill	\makebox[.75in][r]{Date}}}}
% 3. If the document is not to be signed, uncomment the RENEWcommand below
%\renewcommand{\NameSigPair}[1]{#1}

%%%%%%%%%%%%%%%%%%%%%%%%%%%%%%%%%%%%%%%
\begin{document}
\begin{titlepage}
    \pagenumbering{gobble}
    \begin{singlespace}
    	%\includegraphics[height=4cm]{coe_v_spot1}
        \hfill 
        % 4. If you have a logo, use this includegraphics command to put it on the coversheet.
        %\includegraphics[height=4cm]{CompanyLogo}   
        \par\vspace{.2in}
        \centering
        \scshape{
            \huge CS Capstone \DocType \par
            {\large\today}\par
            \vspace{.5in}
            \textbf{\Huge\CapstoneProjectName}\par
            \vfill
            {\large Prepared for}\par
            \Huge \CapstoneSponsorCompany\par
            \vspace{5pt}
            {\Large\CapstoneSponsorPerson\par}
            {\large Prepared by }\par
            Group\CapstoneTeamNumber\par
            % 5. comment out the line below this one if you do not wish to name your team
            \CapstoneTeamName\par 
            \vspace{5pt}
            {\Large
                \GroupMemberOne\par
                \GroupMemberTwo\par
                \GroupMemberThree\par
            }
            \vspace{20pt}
        }
        
        
       % \pagebreak
        \begin{abstract}
        % 6. Fill in your abstract    
        	This is our operations guide on how to run our A-Frame Livestream Portal project. This document will explain how to navigate through the website to view streams it will also describe how to connect the streaming devices. The document however does not explain how to install the software for the website because it is out of this documents scope.
         \end{abstract} 
     \end{singlespace}
\end{titlepage}   

    

\newpage
\pagenumbering{arabic}
\tableofcontents
% 7. uncomment this (if applicable). Consider adding a page break.
%\listoffigures
%\listoftables
\clearpage

% 8. now you write!
\section{Option 1 - The Viewer}
1)  On the homepage of the website, in the login section enter the token 'viewer' and either press enter or select the icon to the immediate right of the login section. \\
2) To exit the viewer mode and return to the home page, select the logout option on the top right of the screen.

\subsection{View the stream}
1) To view the stream, select any icon that is visible from the venue image. \\
    \hspace*{1cm} a) If it is a regular stream, the viewer can watch the stream as it is displayed on the website. \\
    \hspace*{1cm} b) If it is a 360-degree stream, the viewer can either watch the stream regularly or they can select the view icon. 
    \hspace*{1cm} on the bottom right of the stream to enter the interactive 360-degree streaming mode. \\
2) To exit the stream, select either the back arrow on the top left of the video screen or the logout option on the top right \\ 
\hspace*{1cm} Note: selecting the logout option will return you to the homepage. 

\section{Option 2 - The Admin}
1) On the homepage of the website, in the login section enter the token 'admin' and either press enter or select the icon to the immediate right of the login section. \\
2) To exit the admin mode and return to the home page, select the logout option on the top right of the screen.

\subsection{View the stream}
1) To view the stream, select any icon that is visible from the venue image. \\
    \hspace*{1cm} a) If it is a regular stream, the viewer can watch the stream as it is displayed on the website. \\
    \hspace*{1cm} b) If it is a 360-degree stream, the viewer can either watch the stream regularly or they can select the view icon. \\
2) To exit the stream, select either the back arrow on the top left or the logout option on the top right. \\
\hspace*{1cm} Note: selecting the logout option will return you to the homepage.

\subsection{Add a device}
1) To add a device, select the 'ADD DEVICE' option on the top of the menu. \\
\hspace*{1cm} a) Device type: select the type of device you are adding from the following: \\
    \hspace*{2cm} - regular stream \\
    \hspace*{2cm} - 360-degree stream \\
    \hspace*{2cm} - an audio stream \\
\hspace*{1cm} b) Device name: this is the name you choose for your pi. \\ 
\hspace*{1cm} c) Device Code: this is the code you choose for your pi, please make sure this unique for each pi. \\
    \hspace*{2cm} Note: this will be used internally when the device icon is selected on the website. \\
\hspace*{1cm} d) Rows/Columns: the rows and columns you wish the dimensions of the grid to be. This will help with placing \hspace*{2cm } your pi on the map grid.\\
    \hspace*{2cm} Note: the rows and columns numbers must be positive integers.  \\
\hspace*{1cm} e) IP: enter the IP address of the raspberry pi you are connecting to. \\
\hspace*{2cm} Note: we put in the IP address of the router since we are port forwarding. \\
\hspace*{1cm} f) Port: enter the port number the raspberry pi is streaming to. \\
Note: if there is an error with the format of any of the data entries, an error will pop up to direct the user to change their entry.
Once the desired information is entered correctly, select 'submit' to add the device or 'cancel' to exit the 'ADD DEVICE' window. 

\subsection{Removing a device}
1) To remove a device, select the 'DELETE DEVICE' option on the top of the menu. \\
2) Select the name of the device you wish to delete from the drop-down menu and select 'submit' to remove the device or 'cancel' to exit the 'DELETE DEVICE' window.

\subsection{Move a device}
1) To move a device, select the 'MOVE DEVICE' option on the top of the menu. \\
2) Select the icon of the device you wish to move. \\
\hspace*{1cm} a) All available spaces that the device can be moved to will turn green. \\
3) Select the space you wish to move the device on the map to relocate it.
\\
\hspace*{1cm} Note: if you wish to exit 'MOVE DEVICE' without changing the device locations \\
     \hspace*{1.5cm} select 'CANCEL MOVE' in the menu.

\subsection{Change map}
1) To change the map, select the 'CHANGE MAP' option on the top of the menu. \\
2) Select the 'Choose File' option which brings up a file explorer to choose the file that you wish to replace your current map with. \\
\hspace*{1cm} Note: the file format for the image to be used as a map must be in jpg.
\\
3) Select the dimensions of the map by entering the number of rows and columns as positive integers. \\
\hspace*{1cm} Note: if no file is chosen then the current map's dimensions will be changed instead. \\
4) Select 'UPLOAD FILE' to update the map, or 'CANCEL' to exit.

\section{Conclusion}
This concludes the tutorial for the A-frame live stream portal project.



\end{document}