\documentclass[onecolumn, draftclsnofoot,10pt, compsoc]{IEEEtran}
\usepackage{graphicx}
\usepackage{url}
\usepackage{setspace}

\usepackage{geometry}
\geometry{textheight=9.5in, textwidth=7in}

% 1. Fill in these details
\def \CapstoneTeamName{		Nexusphere}
\def \CapstoneTeamNumber{		48}
\def \GroupMemberOne{			Meghan Mowery}
\def \GroupMemberTwo{			Louis Duvoisin}
\def \GroupMemberThree{			Sarahi Pelayo}
\def \CapstoneProjectName{		A-Frame Live Stream Portal}
\def \CapstoneSponsorCompany{	Oregon State University}
\def \CapstoneSponsorPerson{		Behnam Saeedi}

% 2. Uncomment the appropriate line below so that the document type works
\def \DocType{		Problem Statement
				%Requirements Document
				%Technology Review
				%Design Document
				%Progress Report
				}
			
\newcommand{\NameSigPair}[1]{\par
\makebox[2.75in][r]{#1} \hfil 	\makebox[3.25in]{\makebox[2.25in]{\hrulefill} \hfill		\makebox[.75in]{\hrulefill}}
\par\vspace{-12pt} \textit{\tiny\noindent
\makebox[2.75in]{} \hfil		\makebox[3.25in]{\makebox[2.25in][r]{Signature} \hfill	\makebox[.75in][r]{Date}}}}
% 3. If the document is not to be signed, uncomment the RENEWcommand below
\renewcommand{\NameSigPair}[1]{#1}

%%%%%%%%%%%%%%%%%%%%%%%%%%%%%%%%%%%%%%%
\begin{document}
\begin{titlepage}
    \pagenumbering{gobble}
    \begin{singlespace}
    	%\includegraphics[height=4cm]{coe_v_spot1}
        \hfill 
        % 4. If you have a logo, use this includegraphics command to put it on the coversheet.
        %\includegraphics[height=4cm]{CompanyLogo}   
        \par\vspace{.2in}
        \centering
        \scshape{
            \huge CS Capstone \DocType \par
            {\large\today}\par
            \vspace{.5in}
            \textbf{\Huge\CapstoneProjectName}\par
            \vfill
            {\large Prepared for}\par
            \Huge \CapstoneSponsorCompany\par
            \vspace{5pt}
            {\Large\NameSigPair{\CapstoneSponsorPerson}\par}
            {\large Prepared by }\par
            Group\CapstoneTeamNumber\par
            % 5. comment out the line below this one if you do not wish to name your team
            \CapstoneTeamName\par 
            \vspace{5pt}
            {\Large
                \NameSigPair{\GroupMemberOne}\par
                \NameSigPair{\GroupMemberTwo}\par
                \NameSigPair{\GroupMemberThree}\par
            }
            \vspace{20pt}
        }
        \begin{abstract}
        % 6. Fill in your abstract    
        	A-Frame Live Stream Portal is a project that will be used to bring families closer together, even when adversity keeps them apart. This project will contain elements of both software and hardware development.We will create a solution that will be implemented into a system which creates an immersive experience in real time. Our sponsors, Behnam Sadeedi and Jenna Shearer, who will be getting married would like for the groom’s family and relatives who cannot physically travel to the United States to still be a part of the ceremony. Users will be able to tune into the interactive website,  which has a a map of the venue, with or without VR headsets. They will be able to choose a device on the map and view the live stream from a camera near that point. A system of this nature will bring families together by sharing special moments they might have otherwise missed. 
        \end{abstract}     
    \end{singlespace}
\end{titlepage}
\newpage
\pagenumbering{arabic}
\tableofcontents
% 7. uncomment this (if applicable). Consider adding a page break.
%\listoffigures
%\listoftables
\clearpage

% 8. now you write!
\section{Problem}
\IEEEPARstart{T}{he} supreme court decided in a 5-to-4 vote to support President Trump’s travel ban and restrictions on people from Iran, Libya, North Korea, Somalia, Syria, Venezuela and Yemen traveling into the United States of America [1]. 
It is estimated that one million Iran American citizens live in the United States. 
The number is so high because Iran produces more visas than the other countries. 
The issue now comes when relatives in Iran want to visit their families in America.
Jamal Abdi, vice president of policy at the national Iranian American Council, points out that, to travel, Iranians must go through an unpredictable process to obtain a waiver [1]. 
The uncertainty of obtaining the waiver makes it almost impossible for Iranian people to travel to the United States for the time being. 
Behman Seedi's family would like to attend his wedding, however their travel plans are disrupted by the improbable chance of getting past the travel ban. 
To accommodate for them not being there physically, our sponsor wishes to create a system where the family can still view the wedding.
The main problem that we will be solving is giving an immersive, and complete experience while also considering distance, and efficiency.
Since the sponsor's family lives in another country, we will need to research possible issues with bandwidth, delay during live streaming, and time-zone differences.
We also need the solution to be efficient or else the family might miss an important part of the wedding due to a delay in the streaming.
Another problem that will have to be solved is the possibility of a delay between audio and visual output. 
This delay would make the experience not as enjoyable for a viewer as it could be challenging to understand what is happening during the ceremony if the audio is not in sync with the video.
To solve this problem, the sponsor suggests that we create an interactive web portal. 
Wedding preparation is stressful, so we will need to make the project simple enough so the sponsor is not stressed about the project. 
\newline 

\section{Solution}
Our desire is to create a system that is more interactive than traditional forms of wedding videography and photography.
Normal wedding pictures and video need to be edited which can take days and relatives will have missed the events taking place.
To include our sponsor's absent family, the wedding will instead be streamed live. 
The solution for this problem is combining a web portal with various cameras that the relatives can select which will allow them to view the proceedings.
We will create an attractive website with an intuitive interface which will allow those tuning in to explore the venue using a complete map by clicking on the device icons. 
The map must also have an edit mode which gives it the capability of being updated, letting the camera's locations be moved to different parts of the venue, as well as giving it the option to add or remove cameras at will.
The web portal will need to be simple and easy to understand so usability does not become an issue for the family viewing the wedding.
The web portal will be developed using Linux as the operating system, Apaches as the web server, MySQL as the relational database, PHP as the scripting language, and Javascript in conjunction with a Mozilla A-Frame as a framework for handling the 360 video stream. 
\newline
\newline

Another part of the solution will be the cameras themselves; we want to make them sturdy to reduce the risk of them falling over and breaking. 
The cameras will broadcast to a website that will be streaming everything in real time at a minimum of three locations. 
One of the cameras will also have 360 degree video streaming capability that will provide the relatives an even more immersive experience than looking at wedding photos days later.  
The streaming devices will be positioned around the venue and will have their own internet connection so that they can stream their video feed.
Therefore these cameras will also need their own WiFi modules and work independently.
The cameras will have at least 1080p at 30fps stream quality, they will also need to be portable, and easy to set up.
The ideal choice would then be to use batteries to allow for ease of use and, being portable, we would not be able to have any type of wires connected to them.
The batteries will either need to be easily swappable, or be large enough to power the cameras for the entire duration of the wedding ceremony.
Our hope is that the family will have an enjoyable experience that they would not be able to have in other circumstances.

\section{Performance Metrics}
The complete project will be separated into two parts, one hardware and one software.
To capture the wedding experience a minimum of three portable streaming devices will be deployed at the venue. 
Our project will be considered complete when we are able to connect at least 3 cameras to our server, have them stream video to the server, and have the server hosting a website that users can connect to and select a stream to view. 
The website's map will have the physical location of the devices as clickable icons linking to a separate page where users are able to view the stream. 
When a device is clicked a 1080p and 30 frame per second stream will be opened in the new page. 
The map’s interface will have an edit mode that allows for the update of the venue map image using a .png or .jpeg. 
Because the streaming devices are independent, the edit mode will also have the options to add, remove, and reposition devices on the map which can be saved. 
All of the cameras will need to be able to record at 1080p and 30 frames per second, and one of the cameras will also need to broadcast in 360 degree video.
The cameras and microphone will be portable and easy to set up so they can be moved to and from sites as well as within the venue. 
All of the cameras will be independent from one another and be broadcasting to our web portal though the use of independent WiFi modules. 
\\
\\
The project itself will need to be fully functional and be usable by the sponsor and their family, and this means that we will have to complete various tests throughout the creation of the project to make sure that the finished product meets the sponsor's standards.
To check that the product meets the sponsor's standards, we will be conducting usability tests with Jenna after the project is running to make sure the interface is accessible to those with less technical expertise. 
We most likely will include a test run with the family so we can make sure the product is tested as it would be used on the wedding day. 
We hope to also make the cameras themselves visually attractive so they do not take away from the overall appearance of the wedding. 
If we complete all of the necessary parameters with enough time for the deadline, we will work to make the experience even more enjoyable for the family. 
These performance metrics depend on the desire of the sponsor and what they wish for us to create so their entire family can experience and enjoy this special event. 

\begin{thebibliography}{1}

\bibitem{IEEEhowto:kopka}
 Liptak, A. and Shear, M. (2018). Trump’s Travel Ban Is Upheld by Supreme Court. [online] Nytimes.com. Available at: https://www.nytimes.com/2018/06/26/us/politics/supreme-court-trump-travel-ban.html?module=inline [Accessed 12 Oct. 2018].

\end{thebibliography}

\end{document}
